\chapter{\LaTeX 基础介绍}
LaTeX作为专门用于制作文档的计算机程序语言,其语法规则是由命令 (command) 和环
境 (environment) 构成,基于一些基本命令,如\textbackslash usepackage\{\},我们能调用制
作文档所需的宏包。一般而言,在制作文档时,LaTeX的代码结构分为前导 (preamble) 和主
体 (body) 两部分,前导部分的代码主要用于申明文档类型、排版样式、所使用的宏包等;主体部
分主要用于确定标题、章节、目录等文章结构及创作文档内容。

掌握一些常用命令之后,我们便可上手编辑简单的文档。从功能上来说,LaTeX能满足人们对文档制
作多样性的要求。我们可以用LaTeX制作各类文档,包括科技文档、技术报告、学位论文、书籍著
作、幻灯片、个人简历、信件以及海报。不同的文档类型在文档大小、排版、章节样式方面略有不
同,所以在使用LaTeX制作文档时,我们首先需要对文档类型进行申明,然后再进行文档内容的创作。

本章节要介绍的内容主要包括:Latex语法规则、Latex代码结构、文档类型的介绍、简单文档的制
作以及全局格式的设置。

\section{\LaTeX 语法规则}
对于任何一种计算机程序语言,严格的语法规则都是让程序语言实现功能的保障。不同于一般性的计
算机程序语言,LaTeX的语法规则非常简单,是由命令和环境构成。在LaTeX中,不管是命令还是环
境,都离不开计算机符号反斜线\textbackslash,例如调用宏包会用到
\textbackslash usepackage\{宏包名称\},这里的花括号\{\}也是非常常用的一个符号。在调用宏包
后,命令的用法都是由宏包约定的。

\subsection{命令}
LaTeX中有很多命令,它们用法大同小异,功能却千差万别。既有申明文档类型的命令,如\textbackslash documentclass\{article\},
也有输入特殊符号的命令,例如\textbackslash copyright。一般而言,LaTeX中的命令通常由三
部分组成\textbackslash 命令名[可省略参数]\{不可省略参数\},具有以下特点:
\begin{itemize}
    \item 通常都是以反斜线作为开始,通过紧跟的既定字符(命令名)实现相应的功能,例
          如\textbackslash LaTeX和\textbackslash copyright可生成特殊字符。
    \item 一些命令需要申明一些参数,通过大括号中的不可省略参数进行申明,例如\textbackslash color\{blue\}
          命令中需要申明具体的颜色名称。
    \item 一些命令包括一些选项和一些参数一般情况下是默认的,有需要时可以通过中括号中的
          可省略参数进行调整,例如在\textbackslash documentclass[a4paper]\{article\}中,中
          括号[]作为额外的选项,既可以自行设置,也可以选择默认设置。
    \item 有些命令可以用反斜线作为终止,例如\textbackslash copyright。
\end{itemize}

\subsection{环境}
这里所说的“环境”是指编译环境,使用LaTeX时,当我们想要编译出期望的效果,例如列表、插入图
形、制作表格,我们就需要用到一些环境。举例来说,我们可以通过
\begin{lstlisting}[language=TeX]
    \begin{itemize}
        \item item 1 % 条目1
        \item item 2 % 条目2
    \end{itemize}
\end{lstlisting}

制作一个简单的无序列表,这里的\emph{itemize}表示无序列表环境,\emph{\textbackslash begin\{\}}
和\emph{\textbackslash end\{\}}表示环境的初始化和结束。当然,这些环境并非“一成不变”,我
们也可以设置一些参数,从而改变编译之后的文档效果,例如,可以通过
\begin{lstlisting}[language=TeX]
    \begin{spacing}{1.3}
        paragraph 1 % 第1段

        paragraph 2 % 第2段
    \end{spacing}
\end{lstlisting}
将两段话之间的行间距设置为1.3倍。

\emph{【例】}使用基本命令\textbackslash documentclass{article}和文档环境\textbackslash begin\{document\}
和\textbackslash end\{document\}创建一个简单文档。
\begin{lstlisting}[language=TeX]
    \documentclass{article}

    \begin{document}

    Hello, LaTeXers! This is our first LaTeX document.

    \end{document}
\end{lstlisting}

\subsection{宏包}
宏包是LaTeX的重要组成部分,用来扩展和增强LaTeX的功能,是支撑LaTeX实现一系列复杂文档编辑
和排版的关键所在。宏包与LaTeX的关系和浏览器插件与浏览器的关系类似,通过调用不同的宏包可
以实现一些复杂排版功能,例如插入表格、插入公式和特殊符号、插入程序源代码、设置文档样式等。
一个宏包通常会提供一组LaTeX命令,有些特殊命令只能在调用宏包后使用。在LaTeX中,调用宏包的
形式大同小异,如果想要使用某一特定宏包,最简单的办法就是用\textbackslash usepackage\{宏包名\}
命令对相应宏包进行调用。

\emph{【例】}使用\textbackslash usepackage\{color\}命令调用颜色宏包调整文本字体颜色。
\begin{lstlisting}[language=TeX]
    \documentclass{article}
    \usepackage{color} % 调用颜色宏包
    \begin{document}
    \textcolor[rgb]{1,0,0}{Hello, LaTeXers! This is our first LaTeX document.}
    \end{document}
\end{lstlisting}

\section{\LaTeX 代码结构}
在使用LaTeX进行文档编辑时,我们通常在拓展名为\emph{.tex}的源文件中书写代码,然后通过编译,生成
一个PDF格式的文档。LaTeX源文件的代码结构主要包含两个部分,即前导代码(preamble)和主体
代码(body),其结构示例如下:
\begin{lstlisting}[language=TeX]
    \documentclass[]{}

    ...... % 前导代码(preamble)

    \begin{document}

    ...... % 主体代码(body)

    \end{document}
\end{lstlisting}

\subsection{前导代码}
前导代码是指从源文件第一行代码到\textbackslash begin\{document\}之间的所有命令语句,
一般为LaTeX代码的第一部分。在前导代码部分,既可设置文档类型全局参数,包括字体大小、纸张
大小、文字分栏、单双面打印设置等,也可声明主体代码中需要用到的宏包,如插入图形、新增表格
所要用到的宏包。当全局格式没有特殊申明时,前导代码中的文档类型申明语句可以简写成\textbackslash documentclass\{B\},
其中,位置B的作用在于申明文档类型,如article(常规文档)、book(书籍)、report(报告)、
beamer(幻灯片)等。

下面以常规文档(article)为例,简要介绍各常用全局参数的设置方式。

\subsubsection{字体大小}
article类型文档字体大小默认为\emph{10pt},可在\textbackslash documentclass[可省参数]\{article\}
的中括号中根据需要设置成11pt和12pt。

通常,我们按照\textbackslash documentclass[fontsize = 12pt]\{article\}这样的形式设
置文档参数,有时候,为了方便起见,可以把文档类型申明做一定的简写,如\textbackslash documentclass[12pt]\{article\}。
需要注意的是,字体大小设置中的基本单位pt是英文单词point的缩写,是一个物理长度单位,
\textbf{指的是72分之一英寸,即1pt等于1/72英寸}。

\subsubsection{纸张大小}
article类型文档纸张大小默认为\emph{letterpaper},同样可在\textbackslash documentclass
    [可省参数]\{article\}的中括号中根据需要设置成\emph{a4paper}、\emph{a5paper}、\emph{b5paper}、
\emph{legalpaper}和\emph{executivepaper}。

\subsubsection{文字分栏}
article类型文档的文字分栏默认为\emph{onecolumn}(不分栏),也可以使用\emph{twocolumn}
参数设置为twocolumn(两栏)。

\subsubsection{单双面打印设置}
article类型文档打印时默认单面打印,同样可以使用\textbackslash documentclass[可省参数]
\{article\}中括号中的可选参数,通过添加\emph{twoside}参数进行双面打印的设置。

\begin{tcolorbox}[colback=red!5!white, colframe=red!50!black, title=文档属性设置小结]
    基于上面介绍可供调整的参数,我们可以进行任意无序组合,例如\textbackslash documentclass
        [a4paper, 11pt, twoside]\{article\}对应着文档类型为article、纸张大小为A4、字体大小为11pt的双面文档。
\end{tcolorbox}

\subsection{主体代码}
主体代码为\textbackslash begin\{document\}及\textbackslash end\{document\}之间所有
的命令语句和文本,一般由文档的创作内容构成,按照一般书写顺序,主要包含文档标题、目录、
章节、图表及具体文字内容等,通常配合一些基本命令的使用,来形成我们期望的文档。下面给出了
一个简单例子,让读者对如何制作一个包含标题、章节及其文字内容的简单文档能有一个比较粗略的
认识,更具体的语法命令我们将在后续章节中依次介绍。
\begin{lstlisting}[language=TeX]
    \documentclass{article}
    \title{LaTeX cook-book}

    \begin{document}
    \maketitle
    \section{Introduction}

    Hello, LaTeXers! This is our first LaTeX document.

    \end{document}
\end{lstlisting}

\section{文档类型介绍}
在LaTeX代码结构中,申明文档类型往往是制作文档的第一步,也是最基本的一步。事实上,不同
文档类型对应的文档样式略有不同,但制作不同类型的文档时,LaTeX中的绝大多数命令和环境却是
通用的,完成对文档内容的创作后,使用文档类型的申明语句可以让我们在不同类型的文档间切换自如。

\subsection{基本介绍}
对LaTeX熟悉的读者会知道,LaTeX实际上支持非常多的文档类型,例如,撰写科技论文会用到的
\emph{article}(常规文档)类型、制作演示文稿会用到的\emph{beamer}(幻灯片)类型。在众多文档类型中,
常见的文档类型包括\emph{article}(常规文档)、\emph{report}(报告)、\emph{book}(书籍)、\emph{beamer}(幻灯片)等,
如果使用支持中文编译的\emph{ctex}文档类型,还会有\emph{ctexart}(中文常规文档)、\emph{ctexrep}(中文报告)、
\emph{ctexbook}(中文书籍)等,文档类型会直接决定整个文档的样式和风格。

使用LaTeX制作文档时,申明文档类型是作为前导代码,其一般格式为:
\begin{lstlisting}
    \documentclass[A]{B}
\end{lstlisting}

在这一申明语句中,位置A的作用主要是设置控制全文的文档参数,我们可以调整全文的字体大小、
纸张大小、分栏设置等,因为各种文档类型都有一整套的默认参数,所以一般情况可以省略掉位置A。
在位置B,我们需要键入特定的文档类型,例如,\textbackslash documentclass[a4paper, 12pt]\{article\}
即表示申明一个纸张大小为A4、字体大小为12pt的常规文档。

以下将逐一介绍比较常用的三种文档类型,包括article(常规文档)、report(报告)、book(书籍)。
其中,report和book这两种文档类型的文档结构是一致的,可以使用的结构命令有\textbackslash part\{\}、
\textbackslash chapter\{\}、\textbackslash section\{\}、\textbackslash subsection\{\}、
\textbackslash subsubsection\{\}、\textbackslash paragraph\{\}、\textbackslash subparagraph\{\},
举例来说,包含颜色命令的而article文档类型中除了没有\textbackslash chapter\{\}这一结构命令
之外,其他都与report和book文档类型是一样的。

\begin{itemize}
    \item article是LaTeX制作文档时最为常用的一种文档类型,撰写科技论文往往会用到article文档类型。
    \item report主要是面向撰写各类技术报告的文档类型。
    \item book是用于制作书籍等出版物的文档类型。
\end{itemize}

\section{简单文档的制作}
LaTeX不但适合制作篇幅较大的文档,在制作篇幅较小的文档比如手稿、作业等时也十分方便。在LaTeX
的各类文档中,最为常用的文档类型为article (文章),以下将介绍如何制作一个简单文档。

\subsection{制作封面}
添加标题、日期、作者信息一般是在\textbackslash begin\{document\}之前,格式如下:
\begin{lstlisting}
    % 输入空格表示空的
    \title{标题}
    \author{作者名字}
    \date{日期} % 如果不设置则会自动设置为编译时的时间,如果不想展示日期则使用\data{}
\end{lstlisting}

如果要显示添加的相关信息,需要在\textbackslash begin\{document\}之后使用\textbackslash maketitle命令。

\subsection{开始创建文档}
在LaTeX中,以\textbackslash begin\{document\}命令为分界线,该命令之前的代码都统称为
前导代码,这些代码能设置全局参数。位于\textbackslash begin\{document\}和
\textbackslash end\{document\}之间的代码被视为主体代码,我们所创作文档的具
体内容也都是放在这两个命令之间。

\subsubsection{设置章节}
文档的章节是文档逻辑关系的重要体现,无论是中文论文还是英文论文都会有严谨的格式,章、节、段分明。
在LaTeX中,不同的文档类设置章节的命令有些许差别,\textbackslash chapter命令只在
book、report两个文档类中有定义,article类型中设置章节可以通过\textbackslash section\{name\}
及\textbackslash subsection\{name\}等简单的命令进行实现。

\subsubsection{段落}
段落是文章的基础,在LaTeX中,可以之间在文档中间键入空行文本作为段落,也可以使用
\textbackslash paragraph\{name\}和\textbackslash subparagraph\{name\}插入带标题的
段落和亚段落。

\subsubsection{生成目录}
在LaTeX中,我们可以通过一行简单的命令便可以生成文档的目录,即\textbackslash tableofcontents。
命令放在哪里,就会在哪里自动创建一个目录。默认情况下,该命令会根据用户定义的篇章节标题生
成文档目录。目录中包含\textbackslash subsubsection及其更高层次的结构标题,而段落和子段
信息则不会出现在文档目录中。注意如果有带\emph{*}号的章节命令,则该章节标题也不会出现在
目录中。如果想让文档正文内容与目录不在同一页,可在\textbackslash tableofcontents命令
后使用\textbackslash newpage命令或者\textbackslash clearpage命令。

类似对章节编号深度的设置,我们通过调用计数器命令\textbackslash mintinline\{tex\}\{\textbackslash setcounter\}
也可以指定目录层次深度。例如:
\begin{itemize}
    \item \textbackslash setcounter\{tocdepth\}\{0\} 目录层次仅包括part
    \item \textbackslash setcounter\{tocdepth\}\{1\} 目录层次深入到section
    \item \textbackslash setcounter\{tocdepth\}\{2\} 目录层次深入到subsection
    \item \textbackslash setcounter\{tocdepth\}\{3\} 目录层次深入到subsubsection,默认值
\end{itemize}

除此之外,我们还可以在章节前面添加\textbackslash addtocontents\{toc\}\{\textbackslash setcounter\{tocdepth\}\{\}\}
命令对每个章节设置不同深度的目录。另外还有一些其他的目录格式调整命令,如果我们想让创建的
目录在文档中独占一页,只需要在目录生成命令前后添加\textbackslash newpage;如果我们需要
让目录页面不带有全文格式,只需要在生成目录命令后面加上\textbackslash thispagestyle\{empty\}
命令;如果我们想设置目录页之后设置页码为1,则需要在生成目录命令后面加上\textbackslash setcounter\{page\}\{1\}
命令。

如果我们想要创建图目录或表目录,分别使用\textbackslash listoffigures、\textbackslash listoftables
命令即可,与创建章节目录的过程类似,这两个命令会根据文档中图表的标题产生图表目录,但不同
之处在于,图目录或表目录中所有标题均属于同一层次。

\emph{【例】}基于当前已掌握的知识的一个简单的article:
\lstinputlisting[language=TeX]{code/demo_1.tex}

\section{一些基本命令}
当我们运用LateX进行文档编辑时,需要用到一些基本命令。

\subsection{全局格式设置}
在前面,我们介绍了一些全局设置的命令在申明文档类型时可以进行的一些全局参数设置,使用方法
为在\textbackslash documentclass[可省参数]\{article\}的中括号中根据需要进行参数设置,
然而有些全局设置需要用到一些其他方法进行调整。例如,纸张方向、页边距等需要调用宏包进行参
数调整。同样地,我们以article类型文档进行举例说明。

\subsubsection{纸张方向}
article类型文档的纸张方向默认为\emph{portrait}(纵向),也可以设置成\emph{landscape}(横向)。
在文档中可以调用\emph{lscape}宏包中的\textbackslash begin\{landscape\} \textbackslash end\{landscape\}
环境将默认的纵向文档调整为横向。

\begin{tcolorbox}[colback=red!5!white, colframe=red!50!black,
        title=How to change certain pages into landscape/portrait mode]
    https://tex.stackexchange.com/q/337/227605
\end{tcolorbox}

\subsubsection{页边距}
article类型文档的页边距可以通过调用\emph{geometry}宏包进行调整
\begin{lstlisting}[language=TeX]
    \usepackage{geometry} % 使用页面设置宏包
    \geometry{left=3.0cm,right=3.0cm,top=2.5cm,bottom=2.5cm} % 设置页边距
\end{lstlisting}

\subsubsection{章节标题的字体格式}
article类型文档的章节标题的字体格式可以通过调用\emph{sectsty}宏包进行调整。

一些设置例举:
\begin{lstlisting}
    \usepackage{sectsty}
    \sectionfont{\fontfamily{phv}\fontseries{b}\fontsize{11pt}{20pt}\selectfont} % 一级标题字体格式设置
    \subsectionfont{\fontfamily{phv}\fontseries{b}\fontsize{11pt}{20pt}\selectfont} % 二级标题字体格式设置
    \subsubsectionfont{\fontfamily{phv}\fontseries{b}\fontsize{11pt}{20pt}\selectfont} % 三级标题字体格式设置
\end{lstlisting}

\subsubsection{图、表、公式格式全局设置}
当我们需要批量设置图、表及公式的格式时,可以通过调用caption宏包进行全局设置。
例:
\begin{lstlisting}[language=TeX]
    \usepackage[labelfont=bf,labelsep=period,font={bf,sf,normalsize}]{caption}
\end{lstlisting}

\begin{tcolorbox}[colback=red!5!white, colframe=red!50!black,
        title=Changing/Defining Fonts for an Entire Document]
    https://tex.stackexchange.com/q/337/227605
\end{tcolorbox}

\subsubsection{自定义命令全局设置}
有时,我们需要也可以使用一些自定义命令来更改全局设置。例如在更改整个文档的字体格式时,
我们也可以使用:
\begin{lstlisting}[language=TeX]
    \renewcommand{\sfdefault}{\fontencoding{T1}\fontfamily{phv}\selectfont}
    \renewcommand{\familydefault}{\sfdefault}
\end{lstlisting}
等命令。

在更改目录标题时,我们可以使用:
\begin{lstlisting}[language=TeX]
    \renewcommand{\contentsname}{new name of Contents}
\end{lstlisting}