\chapter*{摘要}

最近几年挑战只使用正版免费、或者开源软件,已经把系统替换成Manjaro使用了3年有余。但是开源的办公套件LibreOffice的体验实在不好,偶然在网上看到\LaTeX 的介绍,感觉是个不错的选择。其实TeX、\LaTeX 非常久远,比MicroOffice更早,而且免费。缺点就是需要学习一些相关的命令和宏包的使用。

目前国内大部分毕业论文模板还是使用Word的,但部分院校的硕士、博士论文模板也开始提供\LaTeX 模板。而国际期刊依然使用\LaTeX 作为主流排版软件(Word的别人不一定收)。\LaTeX 可以排版书籍(book)、普通文章(article)、学术报告(report),也可以用来写幻灯片(beamer)。

目前各大视频网站都有\LaTeX 的视频教程,本文是通过github上的latex-cookbook\footnote{该教程需要安装jupyter notebook, https://github.com/xinychen/latex-cookbook}项目学习\LaTeX ,内容也几乎相同,大家可以去该项目学习。本文仅记录文字、表格、公式、插图、图形绘制及幻灯片制作的基本使用。

\section*{关于环境搭建}

大家可以根据自己的系统去网上搜相关的搭建教程,过程并不复杂,不过安装包有点大。另外推荐一个在线\LaTeX 编辑网站,可以随写随编译。

\begin{tcolorbox}[colback=red!5!white, colframe=red!50!black,
        title=Overleaf\, Online LaTeX Editor]
    https://www.overleaf.com/
\end{tcolorbox}

知名科普UP主——妈咪说MommyTalk,也开发了一个在线\LaTeX 公式编辑网站,非常优秀,推荐给大家。

\begin{tcolorbox}[colback=red!5!white, colframe=red!50!black,
        title=在线LaTeX公式编辑器]
    https://www.latexlive.com/
\end{tcolorbox}