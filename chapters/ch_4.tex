\chapter{公式编辑}
在一些文档特别是科技文档写作的过程中,难免涉及复杂数学公式的编辑工作。Macrosoft office
等办公软件繁琐低效的数学公式编辑问题长期困扰着广大科研工作者。为什么这些办公软件数学公式
编辑效率低下,主要原因是数学公式本身就是复杂多变的,而这些软件采用交互式的操作会让用户使
用起来效率变低。而LaTeX具有简洁又强大的数学公式编辑功能,可以通过调用特定的工具包、使用
一些简单的代码就能生成优雅美观的数学公式,这也是LaTeX深受广大科研工作者喜爱的重要原因之一。

本章将介绍在Latex中进行编写公式编辑的方法和技巧。为了方便读者查用和学习,我们将公式编辑
内容主要分为以下部分:公式编辑的基本介绍、常用的数学符号、希腊字母、微积分中的常用公式、
线性代数中的常用公式、概率论和数理统计、优化理论中的公式。这样分类的目的就是便于读者根据
需要,可以很快查找出合适的公式编辑方法。公式编辑的基本介绍主要是关于数学公式换的设定和基
本格式的调整;而常用的数学符号包括运算符号、标记符号、各类括号、空心符号和特殊函数;希腊
字母主要用于一些数学的变量表示。

而微积分部分主要包括极限、导数和积分;线性代数部分包括矩阵、符合和范数;概率论和数理统计
部分包括概率论基础和概率分布;最后是优化理论部分。以上这些是科研工作者在数学公式编辑时高
频率遇到的公式编辑内容,因此我们将其分类进行介绍,这样可以方便读者提高编辑的效率和质量。

\section{基本介绍}
由于LaTeX编辑的数学公式颜值非常高,很多理工科研究领域的顶级期刊甚至明确要求投稿论文必须
按照给定的LaTeX模板进行论文排版,这样做一方面能保证论文整体排版的美观程度,另一方面也能
让生成出来的数学公式更加规范。一般而言,使用LaTeX编辑公式的一系列规则与数学公式的编写原
则是一致的,例如,在LaTeX中,我们可以用\$\verb|\|frac\{\verb|\|partial f\}\{\verb|\|partial x\}\$
生成偏导数$\frac{\partial f}{\partial x}$。

\subsection{数学公式环境}
\subsubsection{美元符号}
在LaTeX中生成数学公式也有一些基本规则,插入公式的方式有很多种,最基本的一种方式是使用美元
符号,这种方式不仅在LaTeX适用,在Markdown中也是适用的,具体插入数学公式的方法是:
\begin{itemize}
    \item 如果我们想插入行内公式,可以直接在两个美元符号中间编辑需要的公式。
    \item 如果想用美元符号插入行间公式,我们需要输入四个美元符号,与此同时,在四个美元符
          号中间编辑需要的公式。需要注意的是,这里生成的数学公式会自动居中对齐。
\end{itemize}

需要注意的是,LaTeX源文件中的美元符号一般都默认为申明数学公式环境,如果想要在文档中编译出
美元符号,需要在美元符号前加上一个反斜线,这种做法同样适用于百分号,百分号一般被默认为注释功能。

\subsubsection{equation环境}
尽管美元符号可以在行间插入公式,但却没办法对公式进行编号。自动生成带有公式编号的行间公式
需要用到\emph{equation}环境,使用\emph{equation}环境, LaTeX编译时会自动将公式进行居中对齐。

在equation环境中,如果不需要公式编号,只需要在数学公式环境中加上一个星号,例如,
使用\verb|\|begin\{equation*\} \verb|\|end\{equation*\}就可以移除公式编号。

更进一步,在equation环境中,如果想对公式进行索引,可以使用\verb|\|label和\verb|\|eqref两个命令。

\emph{【例】}在数学公式环境中对数学公式进行索引:
\begin{lstlisting}[language=TeX]
    \documentclass[12pt]{article}
    \begin{document}

    Equation~\eqref{eq1} shows a simple formula.

    \begin{equation}\label{eq1}
    x+y=2
    \end{equation}

    \end{document}
\end{lstlisting}

\subsubsection{align环境}
在LaTeX中,除了equation数学公式环境,还有其他几种数学公式环境可以使用。我们要介绍的第一种
是\emph{align}环境,它主要用于数组型的数学表达式,align环境可以将公式进行自动对齐,它也
能对每一条数学表达式分别进行公式编号。

\emph{【例】}使用align环境编译一个方程组:
\begin{lstlisting}[language=TeX]
    \documentclass[12pt]{article}
    \usepackage{amsmath}
    \begin{document}

    %% 使用align环境
    \begin{align}
    x+y=2 \\
    2x+y=3
    \end{align}

    \end{document}
\end{lstlisting}

在align环境中,如果不需要公式编号,同样只需要在数学公式环境中加上一个星号即可,利用这一
特性,我们可以使用align环境,定义多列公式。

需要注意的是,如果想对多行公式对齐,并且共用同一个公式编号,可以在equation环境内使用\emph{aligned}
环境,这里的aligned与align功能类似。

\emph{【例】}在equation内使用aligned编译多列公式:
\begin{lstlisting}[language=TeX]
    \documentclass[12pt]{article}
    \usepackage{amsmath}
    \begin{document}

    \begin{equation}
    \begin{aligned}
    2x+1&=7 & 3y-2&=-5 & -5z+8&=-2 \\
    2x&=6 &   3y&=-3 &   -5z&=-10 \\
    x&=3 &    y&=-1 &     z&=2
    \end{aligned}
    \end{equation}

    \end{document}
\end{lstlisting}

当然,我们也能只对align环境中的某一些公式进行编号,而其他公式不作编号。

\begin{tcolorbox}[colback=red!5!white, colframe=red!50!black,
        title=Eqnarray: numbering last line only.]
    https://latex.org/forum/viewtopic.php?f=46\&t=4543
\end{tcolorbox}

\emph{【例】}使用align编译一个方程组,并且只对第2个方程进行编号:
\begin{lstlisting}[language=TeX]
    \documentclass[12pt]{article}
    \usepackage{amsmath}
    \begin{document}

    \begin{equation}
    \begin{aligned}
    2x+1&=7 & 3y-2&=-5 & -5z+8&=-2 \\
    2x&=6 &   3y&=-3 &   -5z&=-10 \\
    x&=3 &    y&=-1 &     z&=2
    \end{aligned}
    \end{equation}

    \end{document}
\end{lstlisting}

\subsubsection{gather环境}
我们要介绍的第二种数学公式环境是\emph{gather},它既可以将公式进行居中对齐,也能对每一条
数学表达式分别进行公式编号。同样的,如果想要移除公式编号,只需要在公式环境中加上星号即可。

\emph{【例】}使用gather编译一个方程组:
\begin{lstlisting}[language=TeX]
    \documentclass[12pt]{article}
    \usepackage{amsmath}
    \begin{document}

    \begin{gather}
    x+y=2 \\
    2x+y=3
    \end{gather}

    \end{document}
\end{lstlisting}

\subsection{基本格式调整}
\subsubsection{字符类型}
在文本编辑中,我们已经介绍了几种常见的字符类型,实际上,对于数学公式而言,书写时也可以设
置不同的字符类型。以$X,Y,Z,x,y,z$为例,具体而言:
\begin{itemize}
    \item 命令\verb|\|boldsymbol\{X,Y,Z,x,y,z\},编译后的效果为 $\boldsymbol{X,Y,Z,x,y,z}$ ,
          使用之前需申明\verb|\|usepackage\{amsmath\};
    \item 命令\verb|\|mathrm\{X,Y,Z,x,y,z\},编译后的效果为 $\mathrm{X,Y,Z,x,y,z}$ ;
    \item 命令\verb|\|mathit\{X,Y,Z,x,y,z\},编译后的效果为 $\mathit{X,Y,Z,x,y,z}$ ;
    \item 命令\verb|\|mathbf\{X,Y,Z,x,y,z\},编译后的效果为 $\mathbf{X,Y,Z,x,y,z}$ ;
    \item 命令\verb|\|mathsf\{X,Y,Z,x,y,z\},编译后的效果为 $\mathsf{X,Y,Z,x,y,z}$ ;
    \item 命令\verb|\|mathtt\{X,Y,Z,x,y,z\},编译后的效果为 $\mathtt{X,Y,Z,x,y,z}$ ;
    \item 命令\verb|\|boldmath\{X,Y,Z,x,y,z\},编译后的效果为 $\boldmath{X,Y,Z,x,y,z}$ ;
          依赖于特定工具包,使用之前需申明\verb|\|usepackage\{amssymb\};
    \item 命令\verb|\|mathcal\{X,Y,Z\},编译后的效果为 $\mathcal{X,Y,Z}$ ;
    \item 命令\verb|\|mathbb\{X,Y,Z\},依赖于特定工具包,使用之前需申明\verb|\|usepackage\{amssymb, amsfonts\},
          编译后的效果为 $\mathbb{X,Y,Z}$ ,概率论与数理统计中常见的数学期望符号 $\mathbb{E}$
          也是用该命令编译的,即\verb|\|mathbb\{E\};
    \item 命令\verb|\|mathfrak\{X,Y,Z,x,y,z\},依赖于特定工具包,使用之前需申明
          \verb|\|usepackage\{amssymb, amsfonts, eufrak\},编译后的效果为 $\mathfrak{X,Y,Z,x,y,z}$ 。
\end{itemize}

除此之外,如果想在公式中插入正常的文本,可以使用\verb|\|text\{\}命令,例如T\_\{\verb|\|text\{start\}\}
编译后的效果为 $T_{\text{start}}$ 。

\subsubsection{调整公式大小}
如果想对单个公式调整公式字符大小,在美元符号插入的公式中,可以使用\emph{displaystyle}、
\emph{textstyle}、\emph{scriptstyle}和\emph{scriptscriptstyle}等申明命令对公式大小
进行调整,公式显示效果依次从小到大,这些命令一般放在公式前即可。

\emph{【例】}使用上述四种命令分别书写函数 $f(x)=\sum_{i=1}^{n}\frac{1}{x_{i}}$ :
\begin{lstlisting}[language=TeX]
    \documentclass[12pt]{article}
    \begin{document}

    $\displaystyle{f(x)=\sum_{i=1}^{n}\frac{1}{x_{i}}}$, 
    $\textstyle{f(x)=\sum_{i=1}^{n}\frac{1}{x_{i}}}$,
    $\scriptstyle{f(x)=\sum_{i=1}^{n}\frac{1}{x_{i}}}$,
    $\scriptscriptstyle{f(x)=\sum_{i=1}^{n}\frac{1}{x_{i}}}$.

    \end{document}
\end{lstlisting}

在各类公式环境(如equation、align、gather)中,可以外使用一系列字符大小命令进行调整,
例如用\verb|\|begingroup \verb|\|endgroup限定字符区域。

\emph{【例】}在\verb|\|begingroup \verb|\|endgroup中使用字符大小命令\verb|\|small和
\verb|\|Large对公式大小进行调整:
\begin{lstlisting}[language=TeX]
    \documentclass[12pt]{article}
    \usepackage{amsmath}
    \begin{document}

    %% Small size
    \begingroup
    \small
    \begin{align}
    x+y=2 \\
    2x+y=3
    \end{align}
    \endgroup

    %% Large size
    \begingroup
    \Large
    \begin{align}
    x+y=2 \\
    2x+y=3
    \end{align}
    \endgroup

    \end{document}
\end{lstlisting}

\subsubsection{其他格式调整}
在equation、align等公式环境中,我们也可以通过数组\emph{array}环境对数学公式进行对齐:

\emph{【例】}使用array环境编译一个方程组,并添加半边花括号,如:
\begin{lstlisting}[language=TeX]
    \documentclass[12pt]{article}
    \usepackage{amsmath}
    \begin{document}

    \begin{equation}
    \left\{\begin{array}{l}
    x+y=2 \\
    2x+y=3
    \end{array}\right.
    \end{equation}

    \begin{align}
    \left\{\begin{array}{l}
    x+y=2 \\
    2x+y=3
    \end{array}\right.
    \end{align}

    \end{document}
\end{lstlisting}

编译后:
\begin{equation}
    \left\{\begin{array}{l}
        x+y=2 \\
        2x+y=3
    \end{array}\right.
\end{equation}

其中,对齐的方式有l(左侧对齐)、c(居中对齐)和r(右侧对齐)。

\emph{【例】}使用array环境编译公式,并让公式居中对齐:
\begin{lstlisting}[language=TeX]
    \documentclass[12pt]{article}
    \usepackage{amsmath, mathtools}

    \begin{document}

    \begin{equation}
    \begin{array}{c@{\qquad}c}
    A = B + C
    \qquad\Rightarrow
    & D = E - F, \\ \\
    G = H
    \qquad\Rightarrow
    & K = P + Q + M.
    \end{array}
    \end{equation}

    \end{document}
\end{lstlisting}

编译后:
\begin{equation}
    \begin{array}{c@{\qquad}c}
        A = B + C
        \qquad\Rightarrow
         & D = E - F,     \\ \\
        G = H
        \qquad\Rightarrow
         & K = P + Q + M.
    \end{array}
\end{equation}

当公式过长时,还有一些工具包提供的环境可以让公式进行自动跨行,以工具包\emph{breqn}为例,
在使用时,用\verb|\|begin\{dmath\} \verb|\|end\{dmath\}即可。

\section{常用数学符号}
常用数学符号包括运算符号、标记符号、各类括号、空心符号及一些特殊函数。

\subsection{运算符号}
在初等数学中,最基本的运算规则是加减乘除。在LaTeX中,加法符号和减法符号就是\emph{+}和\emph{-};
而乘法符号有两种,第一种是\verb|\|times,对应于符号\emph{×},第二种是\verb|\|cdot,对
应于符号\emph{⋅},除法符号的命令为\verb|\|div。

\begin{itemize}
    \item + 加
    \item - 减
    \item \verb|\|times 叉乘
    \item \verb|\|cdot 点乘
    \item \verb|\|div 除
    \item / 除
\end{itemize}

在加减的基础上命令\verb|\|pm(由plus和minus首字母构成)和\verb|\|mp(由minus和plus首字母构成)
分别对应着符号\emph{$\pm$}和\emph{$\mp$}。与加减乘除同样常用的运算符号还有大于号、小于号等。

\begin{itemize}
    \item x < y 即 $x < y$
    \item x > y 即 $x > y$
    \item x \verb|\|leq y 即 $x \leq y$
    \item x \verb|\|geq y 即 $x \geq y$
    \item x \verb|\|ll y 即 $x \ll y$ 远小于
    \item x \verb|\|gg y 即 $x \gg y$ 远大于
\end{itemize}

对于集合而言:
\begin{itemize}
    \item \verb|\|cap 即 $\cap$
    \item \verb|\|cup 即 $\cup$
    \item \verb|\|supset 即 $\supset$
    \item \verb|\|subset 即 $\subset$
    \item \verb|\|supseteq 即 $\supseteq$
    \item \verb|\|subseteq 即 $\subseteq$
    \item \verb|\|in 即 $\in$
    \item \verb|\|notin 即 $\notin$
\end{itemize}

\subsection{标记符号}
在数学中,还有一些基本数学表达式也非常重要,例如分式、上标、下标。LaTeX中用于书写分数和
分式的基本命令为\verb|\|frac\{分子\}\{分母\},根据场景需要,也可以选用\verb|\|dfrac\{分子\}\{分母\}
和\verb|\|tfrac\{分子\}\{分母\}。

\begin{itemize}
    \item \verb|\|frac\{x+3\}\{y-5\} 即 $\frac{x+3}{y-5}$
    \item \verb|\|dfrac\{x+3\}\{y-5\} 即 $\dfrac{x+3}{y-5}$
    \item \verb|\|tfrac\{x+3\}\{y-5\} 即 $\tfrac{x+3}{y-5}$
    \item x\^\{x+5\} 即 $x^{x+5}$
    \item x\^\{x\^\{3\}+5\} 即 $x^{x^{3}+5}$
    \item x\_\{x+5\} 即 $x_{x+5}$
    \item x\_\{x\_\{3\}+5\} 即 $x_{x_{3}+5}$
    \item x\_\{1\},x\_\{2\},\verb|\|ldots,x\_\{n\} 即 $x_{1},x_{2},\ldots,x_{n}$
\end{itemize}

与上标和下标对应的还有各种“帽子”符号,即字母上面加符号:
\begin{itemize}
    \item \verb|\|hat\{x\} 即 $\hat{x}$
    \item \verb|\|bar\{x\} 即 $\bar{x}$
    \item \verb|\|tilde\{x\} 即 $\tilde{x}$
    \item \verb|\|vec\{x\} 即 $\vec{x}$
    \item \verb|\|dot\{x\} 即 $\dot{x}$
    \item \verb|\|bar\{xy\} 即 $\bar{xy}$
    \item \verb|\|overline\{xy\} 即 $\overline{xy}$
    \item \verb|\|tilde\{xy\} 即 $\tilde{xy}$
    \item \verb|\|widetilde\{xy\} 即 $\widetilde{xy}$
\end{itemize}

\begin{tcolorbox}[colback=red!5!white, colframe=red!50!black,
        title=Tilde over the letter]
    https://latex.org/forum/viewtopic.php?f=48\&t=11388
\end{tcolorbox}

根号同样是数学表达式中的常见符号,在LaTeX中,根号的命令为\verb|\|sqrt\{\},使用默认设置,
生成的表达式为二次方根,如果想要设置为三次方根,则需要调整默认设置,即\verb|\|sqrt[3]\{\},
以此类推,可以设置四次方根等。

\subsection{各类括号}
在数学表达式中,括号的用处和种类都非常多,例如最常见的小括号、中括号、大括号(即花括号)。

\emph{【例】}书写数学表达式 $\left(\frac{1}{y}+1\right)$ 、 $\left[\frac{1}{y}+1\right]$ 和 $\left\{\frac{1}{y}+1\right\}$ :
\begin{lstlisting}[language=TeX]
    $$x\left(\frac{1}{y}+1\right)$$
    $$x\left[\frac{1}{y}+1\right]$$
    $$x\left\{\frac{1}{y}+1\right\}$$
\end{lstlisting}

有时候,由于公式过长等原因,我们也可以在需要分行处插入\verb|\\|将括号内的公式切分成多行。

在这里,我们可以使用一系列命令代替最常用的\verb|\|left和\verb|\|right组合,如\verb|\|bigl
和\verb|\|bigr组合、\verb|\|Bigl和\verb|\|Bigr组合、\verb|\|biggl和\verb|\|biggr组合、
\verb|\|Biggl和\verb|\|Biggr组合来控制括号大小。

\emph{【例】}利用各组合控制括号大小:
\begin{lstlisting}[language=TeX]
    \begin{equation}

    \left(x+y=z \right) \\
    \bigl(x+y=z \bigr) \\
    \Bigl(x+y=z \Bigr) \\
    \biggl(x+y=z \biggr) \\
    \Biggl(x+y=z \Biggr)
    
    \end{equation}        
\end{lstlisting}

编译后:
\begin{equation}
    \left(x+y=z \right) \\
    \bigl(x+y=z \bigr) \\
    \Bigl(x+y=z \Bigr) \\
    \biggl(x+y=z \biggr) \\
    \Biggl(x+y=z \Biggr)
\end{equation}

在数学公式编辑中,除了以上常见的括号,也有一些广义的“括号”。

\emph{【例】}书写数学表达式:
\begin{lstlisting}[language=TeX]
    \begin{equation}
        x\left|\frac{1}{y}+1\right| \\
        x\left\|\frac{1}{y}+1\right\| \\
        \left<\frac{1}{x},\frac{1}{y}\right> \\
        \langle\frac{1}{x},\frac{1}{y}\rangle
    \end{equation}   
\end{lstlisting}

编译后:
\begin{equation}
    x\left|\frac{1}{y}+1\right| \\
    x\left\|\frac{1}{y}+1\right\| \\
    \left<\frac{1}{x},\frac{1}{y}\right> \\
    \langle\frac{1}{x},\frac{1}{y}\rangle
\end{equation}

在这里,使用半边括号,也能书写出导数的表达式。

\emph{【例】}书写导数:
\begin{lstlisting}[language=TeX]
    $$\left.\frac{dy}{dx}\right|_{x=1}$$
\end{lstlisting}

编译后:
$$\left.\frac{dy}{dx}\right|_{x=1}$$

\subsection{空心符号}
在数学表达式中,我们有时候会用到一些约定俗成的空心符号表示集合,这包括:
\begin{itemize}
    \item 空心R符号\(\mathbb{R}\)表示由所有实数构成的集合。
    \item 空心Z符号\(\mathbb{Z}\)表示由所有整数构成的集合。
    \item 空心N符号\(\mathbb{N}\)表示由所有非负整数构成的集合,如果要表示正整数,使用
          符号\(\mathbb{N}_{+}\)即可。
    \item 空心C符号\(\mathbb{C}\)表示由所有复数构成的集合。
\end{itemize}

上面的例子使用了\texttt{\textbackslash{}(\textbackslash{}mathbb\{字母\}\textbackslash{})}
需要注意的是,要想让LaTeX成功编译出这些空心符号,我们需要调用特定的工具包,即\texttt{\textbackslash{}usepackage\{amsfonts\}},一般而言,为了保证公式的编译不出现意外,还会用到其他工具包,即\texttt{\textbackslash{}usepackage\{amsfonts,\ amssymb,\ amsmath\}}。当然,除了这些,还有许多其他符号,这里不再一一赘述。

\emph{【例】}书写表达式 $X\in\mathbb{R}^{m\times n}$ :
\begin{lstlisting}[language=TeX]
    \usepackage{amsfonts}
    \begin{document}
    $$X\in\mathbb{R}^{m\times n}$$
    \end{document}
\end{lstlisting}

\emph{【例】}使用工具包bbold中的\texttt{\textbackslash{}mathbb}命令书写空心的1、2、3、4、5:
\begin{lstlisting}[language=TeX]
    \usepackage{bbold}
    \begin{document}
    $$\mathbb{1},\mathbb{2},\mathbb{3},\mathbb{4},\mathbb{5}$$
    \end{document}
\end{lstlisting}

编译后:
$$\mathbb{1},\mathbb{2},\mathbb{3},\mathbb{4},\mathbb{5}$$

\subsection{特殊函数}
上标很多时候是用来表示变量的幂,例如$x^{2}$表示$x$的平方,由此可以用上标书写出指数函数如$y=x^2$等。
与指数函数对应的一类常用函数被称为对数函数,是指数函数的反函数,LaTeX提供了一些跟对数函数
相关的命令,包括\texttt{\textbackslash{}log}、\texttt{\textbackslash{}ln},
在命令\texttt{\textbackslash{}log}中,我们可以自行设置底数,而命令\texttt{\textbackslash{}ln}
则表示底数为自然数的对数。

有时候,为了简化数学表达式,我们可能会采用求和或者连乘的写法,在LaTeX中,求和符号对应的
命令为\texttt{\textbackslash{}sum\_\{下标\}\^\{上标\}},连乘符号对应的命令为
\texttt{\textbackslash{}prod\_\{下标\}\^\{上标\}}。

\begin{itemize}
    \item \texttt{\textbackslash{}sum\_\{i=1\}\^\{n\}x\_\{i\}} 求和公式即 $\sum_{i=1}^{n}x_{i}$
    \item \texttt{\textbackslash{}prod\_\{i=1\}\^\{n\}x\_\{i\}} 练乘公式即 $\prod_{i=1}^{n}x_{i}$
\end{itemize}

另外,在初等数学的几何中,我们学过了正弦、余弦等,这些在LaTeX都有定义好的命令可供直接使用。

\begin{itemize}
    \item \texttt{y=\textbackslash{}sin(x)} 正弦函数 $y=\sin(x)$
    \item \texttt{y=\textbackslash{}arcsin(x)} 反正弦函数 $y=\arcsin(x)$
    \item \texttt{y=\textbackslash{}cos(x)} 余弦函数 $y=\cos(x)$
    \item \texttt{y=\textbackslash{}arccos(x)} 反余弦函数 $y=\arccos(x)$
    \item \texttt{y=\textbackslash{}tan(x)} 余弦函数 $y=\tan(x)$
    \item \texttt{y=\textbackslash{}arctan(x)} 反余弦函数 $y=\arctan(x)$
\end{itemize}

\section{希腊字母}
我们在初等数学中便已经学习到了一些常用的希腊字母,例如最常见的$\pi$(对应于\texttt{\textbackslash{}pi}),
圆周率$\pi$约等于3.14,圆的面积为$\pi r^2$、周长为$2\pi r$。在几何学中,我们习惯用各种
希腊字母表示度数,如$\alpha$(对应于\texttt{\textbackslash{}alpha})、$\beta$(对应
于\texttt{\textbackslash{}beta})、$\theta$(对应于\texttt{\textbackslash{}theta})、
$\phi$(对应于\texttt{\textbackslash{}phi})、$\psi$(对应于\texttt{\textbackslash{}psi})、
$\varphi$(对应于\texttt{\textbackslash{}varphi}),使用希腊字母既方便,也容易区分于
x,y,z等其他变量。

实际上,这些希腊字母也可以用来作为变量,在概率论与数理统计中常常出现的变量就包括:
\begin{itemize}
    \item 正态分布中的$\mu$(命令为\texttt{\textbackslash{}mu})、$\sigma$(命令为\texttt{\textbackslash{}sigma});
    \item 泊松分布中的$\lambda$(命令为\texttt{\textbackslash{}lambda});
    \item 通常表示自由度的希腊字母为$\nu$ (命令为\texttt{\textbackslash{}nu})。
\end{itemize}

另外,在不等式中经常用到的希腊字母有:
\begin{itemize}
    \item $\delta$ 命令为\texttt{\textbackslash{}delta};
    \item $\epsilon$ 命令为\texttt{\textbackslash{}epsilon};
    \item $\gamma$ 命令为\texttt{\textbackslash{}gamma};
    \item $\eta$ 命令为\texttt{\textbackslash{}eta};
    \item $\kappa$ 命令为\texttt{\textbackslash{}kappa};
    \item $\rho$ 命令为\texttt{\textbackslash{}rho};
    \item $\tau$ 命令为\texttt{\textbackslash{}tau};
    \item $\omega$ 命令为\texttt{\textbackslash{}omega};
\end{itemize}

当然,前面提到的这些希腊字母在用途上并没有严格的界定,很多时候,我们书写数学表达式时可以
根据需要选用适当的希腊字母。

\emph{【例】}书写椭圆$\frac{x^2}{a^2} + \frac{y^2}{b^2} = 1$的面积公式$S=\pi ab$:
\begin{lstlisting}[language=TeX]
    $$S=\pi ab$$ % 椭圆面积公式
\end{lstlisting}

\emph{【例】}书写不等式$a^{\alpha}b^{\beta}\cdots k^{\kappa}l^{\lambda}\leq a\alpha+b\beta+\cdots+k\kappa+l\lambda$:
\begin{lstlisting}[language=TeX]
    $$a^{\alpha}b^{\beta}\cdots k^{\kappa}l^{\lambda}\leq a\alpha+b\beta+\cdots+k\kappa+l\lambda$$
\end{lstlisting}

\emph{【例】}书写不等式$\phi\left(\frac{x_{1}+x_{2}+\cdots+x_{n}}{n}\right)\leq\frac{\phi\left(x_{1}\right)+\phi\left(x_{2}\right)+\cdots+\phi\left(x_{n}\right)}{n}$:
\begin{lstlisting}[language=TeX]
    $$\phi\left(\frac{x_{1}+x_{2}+\cdots+x_{n}}{n}\right)\leq\frac{\phi\left(x_{1}\right)+\phi\left(x_{2}\right)+\cdots+\phi\left(x_{n}\right)}{n}$$
\end{lstlisting}

与英文字母类似的是,有些希腊字母不但有小写字母,还有大写字母,具体如下:
\begin{itemize}
    \item 命令\texttt{\textbackslash{}Gamma}对应于希腊字母$\Gamma$,命令\texttt{\textbackslash{}varGamma}对应于$\varGamma$;
    \item 命令\texttt{\textbackslash{}Delta}对应于希腊字母$\Delta$,命令\texttt{\textbackslash{}varDelta}对应于$\varDelta$;
    \item 命令\texttt{\textbackslash{}Theta}对应于希腊字母$\Theta$,命令\texttt{\textbackslash{}varTheta}对应于$\varTheta$;
    \item 命令\texttt{\textbackslash{}Lambda}对应于希腊字母$\Lambda$,命令\texttt{\textbackslash{}varLambda}对应于$\varLambda$;
    \item 命令\texttt{\textbackslash{}Pi}对应于希腊字母$\Pi$,命令\texttt{\textbackslash{}varPi}对应于$\varPi$;
    \item 命令\texttt{\textbackslash{}Sigma}对应于希腊字母$\Sigma$,命令\texttt{\textbackslash{}varSigma}对应于$\varSigma$;
    \item 命令\texttt{\textbackslash{}Phi}对应于希腊字母$\Phi$,命令\texttt{\textbackslash{}varPhi}对应于$\varPhi$;
    \item 命令\texttt{\textbackslash{}Omega}对应于希腊字母$\Omega$,命令\texttt{\textbackslash{}varOmega}对应于$\varOmega$。
\end{itemize}

从这些大写希腊字母中可以看到:大写希腊字母的命令是将小写希腊字母的命令首字母进行大写,但
这些大写希腊字母与小写希腊字母的区别却不仅仅是尺寸不同;当大写希腊字母作为变量时,可以采用
斜体字。

\emph{【例】}书写$\Delta x+\Delta y$和$(i,j,k)\in\Omega$:
\begin{lstlisting}[language=TeX]
    $$\Delta x+\Delta y$$
    $$(i,j,k)\in\Omega$$
\end{lstlisting}

\section{微积分}
事实上,数学公式的范畴极为广泛,我们所熟知的大学数学课程中,微积分、线性代数、概率论与
数理统计中数学表达式的符号系统均大不相同。本节将主要介绍如何使用LaTeX对微积分中的数学表
达式进行书写和编译。

\subsection{极限}
求极限是整个微积分中的基石,例如$\lim_{x\to 2}x^{2}$ 对应的LaTeX代码为
\$\texttt{\textbackslash{}lim\_\{x\textbackslash{}to 2\}x\^\{2\}}\$。

\emph{【例】}书写以下求极限的表达式:
$$\lim_{x\to-\infty}\frac{3x^{2}-2}{3x-2x^{2}}=\lim_{x\to-\infty}\frac{x^{2}\left(3-\frac{2}{x^{2}}\right)}{x^{2}\left(\frac{3}{x}-2\right)}=\lim_{x\to-\infty}\frac{3-\frac{2}{x^{2}}}{\frac{3}{x}-2}=-\frac{3}{2}$$
\begin{lstlisting}[language=TeX]
    $$\lim_{x\to-\infty}\frac{3x^{2}-2}{3x-2x^{2}}=\lim_{x\to-\infty}\frac{x^{2}\left(3-\frac{2}{x^{2}}\right)}{x^{2}\left(\frac{3}{x}-2\right)}=\lim_{x\to-\infty}\frac{3-\frac{2}{x^{2}}}{\frac{3}{x}-2}=-\frac{3}{2}$$
\end{lstlisting}

\emph{【例】}书写极限$\lim_{\Delta t\to0}\frac{s(t+\Delta t)+s(t)}{\Delta t}$ \& $\displaystyle{\lim_{\Delta t\to0}\frac{s(t+\Delta t)+s(t)}{\Delta t}}$:
\begin{lstlisting}[language=TeX]
    $\lim_{\Delta t\to0}\frac{s(t+\Delta t)+s(t)}{\Delta t}$ \& $\displaystyle{\lim_{\Delta t\to0}\frac{s(t+\Delta t)+s(t)}{\Delta t}}$
\end{lstlisting}

\subsection{导数}
在微积分中,给定函数$𝑓\left(x\right)$后,我们能够将其导数定义为:
$$f^\prime(a)=\lim_{x\to a}\frac{f(x)-f(a)}{x-a}$$

用LaTeX书写这条公式为:
\begin{lstlisting}[language=TeX]
    $$f^\prime(a)=\lim_{x\to a}\frac{f(x)-f(a)}{x-a}$$
\end{lstlisting}

有时候,为了让分数的形式在直观上不显得过大,可以用:
$$f^\prime(a)=\lim\limits_{x\to a}\frac{f(x)-f(a)}{x-a}$$

用LaTeX书写这条公式为:
\begin{lstlisting}[language=TeX]
    $$f^\prime(a)=\lim\limits_{x\to a}\frac{f(x)-f(a)}{x-a}$$
\end{lstlisting}

其中,\texttt{\textbackslash{}lim}和\texttt{\textbackslash{}limits}两个命令需要配合使用。

需要注意的是,\texttt{f\^\textbackslash{}prime(x)}中的\texttt{\textbackslash{}prime}
命令是标准写法。

\emph{【例】}书写导数的定义$f^\prime(x)=\lim_{\Delta x\to 0}\frac{f(x+\Delta x)-f(x)}{\Delta x}$:
\begin{lstlisting}[language=TeX]
    $$f^\prime(x)=\lim_{\Delta x\to 0}\frac{f(x+\Delta x)-f(x)}{\Delta x}$$
\end{lstlisting}

\emph{【例】}书写函数$f\left(x\right)=3x^{2}+2x^{3}+1$的导数$f^\prime(x)=15x^{4}+6x^{2}$:
\begin{lstlisting}[language=TeX]
    $$f^\prime(x)=15x^{4}+6x^{2}$$
\end{lstlisting}

微分在微积分中举足轻重,\texttt{\textbackslash{}mathrm\{d\}}为微分符号$\mathrm{d}$的
命令,一般而言,微分的标准写法为$\frac{\mathrm{d}^{n}}{\mathrm{d}_{x^{n}}}f\left(x\right)$。

\emph{【例】}书写微分$\frac{\mathrm{d}}{\mathrm{d}x}f(x)=15x^{4}+6x^{2}$和$\frac{\mathrm{d}^{2}}{\mathrm{d}^{2}x}f(x)=60x^{3}+12x$:
\begin{lstlisting}[language=TeX]
    $$\frac{\mathrm{d}}{\mathrm{d}x}f(x)=15x^{4}+6x^{2}$$
    $$\frac{\mathrm{d}^{2}}{\mathrm{d}^{2}x}f(x)=60x^{3}+12x$$
\end{lstlisting}

在微积分中,偏微分符号$\partial$的命令为\texttt{\textbackslash{}partial},对于任意函
数$f\left(x,y\right)$,偏微分的标准写法为$\frac{\partial^{n}}{\partial_{x^{n}}}f\left(x,y\right)$
或$\frac{\partial^{n}}{\partial_{y^{n}}}f\left(x,y\right)$。

\emph{【例】}书写函数$f\left(x,y\right)=3x^5y^2+2x^3y+1$的偏微分$\frac{\partial}{\partial x}f(x,y)=15x^{4}y^{2}+6x^{2}y$和$\frac{\partial}{\partial y}f(x,y)=6x^{5}y+2x^{3}$:
\begin{lstlisting}[language=TeX]
    $$\frac{\partial}{\partial x}f(x,y)=15x^{4}y^{2}+6x^{2}y$$
    $$\frac{\partial}{\partial y}f(x,y)=6x^{5}y+2x^{3}$$
\end{lstlisting}

\emph{【例】}书写偏导数$z=\mu\,\frac{\partial y}{\partial x}\bigg|_{x=0}$:
\begin{lstlisting}[language=TeX]
    $$z=\mu\,\frac{\partial y}{\partial x}\bigg|_{x=0}$$
\end{lstlisting}

\subsection{积分}
积分的标准写法为$\int_{a}^{b}f(x)\,\mathrm{d}x$,代码为\texttt{\textbackslash{}int\_\{a\}\^\{b\}f(x)\textbackslash{},\textbackslash{}mathrm\{d\}x},
其中,\texttt{\textbackslash{}int}表示积分,是英文单词integral的缩写形式,使用\texttt{\textbackslash{},}
的目的是引入一个空格。

\emph{【例】}书写积分$\int\frac{\mathrm{d}x}{\sqrt{a^{2}+x^{2}}}=\frac{1}{a}\arcsin\left(\frac{x}{a}\right)+C$
和$\int\tan^{2}x\,\mathrm{d}x=\tan x-x+C$:
\begin{lstlisting}[language=TeX]
    $$\int\frac{\mathrm{d}x}{\sqrt{a^{2}+x^{2}}}=\frac{1}{a}\arcsin\left(\frac{x}{a}\right)+C$$
    $$\int\tan^{2}x\,\mathrm{d}x=\tan x-x+C$$
\end{lstlisting}

\emph{【例】}书写积分$\int_{a}^{b}\left[\lambda_{1}f_{1}(x)+\lambda_{2}f_{2}(x)\right]\,\mathrm{d}x=\lambda_{1}\int_{a}^{b}f_{1}(x)\,\mathrm{d}x+\lambda_{2}\int_{a}^{b}f_{2}(x)\,\mathrm{d}x$
和$\int_{a}^{b}f(x)\,\mathrm{d}x=\int_{a}^{c}f(x)\,\mathrm{d}x+\int_{c}^{b}f(x)\,\mathrm{d}x$:
\begin{lstlisting}[language=TeX]
    $$\int_{a}^{b}\left[\lambda_{1}f_{1}(x)+\lambda_{2}f_{2}(x)\right]\,\mathrm{d}x=\lambda_{1}\int_{a}^{b}f_{1}(x)\,\mathrm{d}x+\lambda_{2}\int_{a}^{b}f_{2}(x)\,\mathrm{d}x$$
    $$\int_{a}^{b}f(x)\,\mathrm{d}x=\int_{a}^{c}f(x)\,\mathrm{d}x+\int_{c}^{b}f(x)\,\mathrm{d}x$$
\end{lstlisting}

\emph{【例】}书写积分:
\begin{equation}
    \begin{aligned}
        V & =2\pi\int_{0}^{1} x\left\{1-(x-1)^{2}\right\}\,\mathrm{d}x     \\
          & =2\pi\int_{0}^{2}\left\{-x^{3}+2 x^{2}\right\}\,\mathrm{d}x    \\
          & =2\pi\left[-\frac{1}{4} x^{4}+\frac{2}{3} x^{3}\right]_{0}^{2} \\
          & =8\pi/3
    \end{aligned}
\end{equation}

\begin{lstlisting}[language=TeX]
    \begin{equation}
        \begin{aligned}
        V&=2\pi\int_{0}^{1} x\left\{1-(x-1)^{2}\right\}\,\mathrm{d}x \\
        &=2\pi\int_{0}^{2}\left\{-x^{3}+2 x^{2}\right\}\,\mathrm{d}x \\
        &=2\pi\left[-\frac{1}{4} x^{4}+\frac{2}{3} x^{3}\right]_{0}^{2} \\
        &=8\pi/3
        \end{aligned}
    \end{equation}
\end{lstlisting}

上述介绍的都是一重积分,在微积分课程中,还有二重积分、三重积分等,对于一重积分,LaTeX提
供的基本命令为\texttt{\textbackslash{}int},二重积分为\texttt{\textbackslash{}iint},
三重积分为\texttt{\textbackslash{}iiint}、四重积分为\texttt{\textbackslash{}iiiint},
当积分为五重或以上时,一般使用\texttt{\textbackslash{}idotsint},即$\idotsint$。

\emph{【例】}书写积分$\iint\limits_{D}f(x,y)\,\mathrm{d}\sigma$
和$\iiint\limits_{\Omega}\left(x^{2}+y^{2}+z^{2}\right)\,\mathrm{d}v$:
\begin{lstlisting}[language=TeX]
    $$\iint\limits_{D}f(x,y)\,\mathrm{d}\sigma$$
    $$\iiint\limits_{\Omega}\left(x^{2}+y^{2}+z^{2}\right)\,\mathrm{d}v$$
\end{lstlisting}

在积分中,有一种特殊的积分符号是在标准的积分符号上加上一个圈,可用来表示计算曲线曲面积分,
即$\oint_{C}f(x)\,\mathrm{d}x+g(y)\,\mathrm{d}y$,代码为:
\begin{lstlisting}[language=TeX]
    $$\oint_{C}f(x)\,\mathrm{d}x+g(y)\,\mathrm{d}y$$
\end{lstlisting}
