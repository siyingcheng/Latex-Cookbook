\chapter*{引言}
1977年,计算机科学家克努斯博士\footnote{Donald E. Knuth,直译名为唐纳德·尔文·克努斯,
    中文名为高德纳,美国计算机科学家,现代计算机科学的先驱人物,在计算机科学及数学领域著
    有多部影响深远的著作,于1974年获得图灵奖。}开发了一款名为TeX的文档排版系统,作为一
种计算机程序语言,它能够专门用于制作各类技术文档,并且对制作包含数学公式的技术文档具
有良好的适用性。克努斯博士开发TeX其实存在一些意外:上世纪70年代,克努斯博士在修改自己的
著作时,由于当时的排版质量差到让他难以容忍,所以他便转而开始思考能否开发出高质量的文档排
版系统。

在使用过程中,TeX制作文档的方式非常特殊,与今天常用的办公软件Word等截然不同,它是完全使
用计算机程序语言来制作文档的。由于其对计算机语言的高度依赖,这款系统的使用门槛较高,但也
具有很多优点,其中最为人称道的优点是它可以书写大量复杂的数学表达式。基于TeX,兰波特博士
\footnote{Leslie Lamport,
    直译名为莱斯利·兰波特,美国计算机科学家,于2013年获得图灵奖,他获得图灵奖的原因并非
    在于开发了LaTeX,而是源于他在所研究的学术领域做出的突出贡献。}于1985年开发了另一
款文档排版系统,名为\LaTeX ,兰波特博士设计这款系统初衷是让人们从排版样式这些繁琐的
细节中解放出来,从而将精力集中在文档结构和文档内容上,这一做法很快便让LaTeX取代了TeX。
后来,LaTeX的众多开发者对LaTeX最初版本进行了更新和提升,也就是我们今天一直在用的LaTeX。