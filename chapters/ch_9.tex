\chapter{幻灯片制作}

2003年,柏林工业大学的博士生Till Tantau发布了一款用于制作演示文稿的工具Beamer。Beamer
是Till Tantau利用业余时间开发的,他的初衷是方便自己使用LaTeX制作幻灯片,不过出乎意料的是,
后来Beamer的受欢迎程度完全超出了他的想象。在Till Tantau开发Beamer期间,他收到了很多人的
建议和反馈,这些都直接推动了开发工作。2010年,Till Tantau将Beamer交由他人维护和管理。

Beamer作为LaTeX的一种特殊文档类型,它的出现无疑丰富了LaTeX制作演示文稿的功能。虽然Beamer
并非LaTeX中第一款用于制作演示文稿的工具,但直到今日,Beamer却是最受大家欢迎的一款。Beamer
的使用方式简单灵活,只需设定LaTeX文档类型为beamer,便可开始创作。同时,Beamer提供了大量
的幻灯片模板,这些模板包含了丰富多彩的视觉效果,创作者可以直接使用。毫不夸张地说,Beamer
的出现极大地提高了人们使用LaTeX制作幻灯片的热情。值得一提的是,在2005年,Till Tantau又
发布了一款非常便捷的LaTeX绘图工具TikZ。TikZ不仅可以辅助Beamer幻灯片的制作,也可以应用于
科技文档中的各类绘图任务。

Beamer是随着LaTeX的发展而衍生出来的一种特殊文档类型,也常常被看作是一个功能强大的宏包,
可以支撑科研工作者制作幻灯片的需求。使用Beamer制作幻灯片与Office办公软件(如PowerPoint)
完全不同,从本质上来说,使用Beamer制作幻灯片其实和LaTeX中的其他文档类型没有太大区别:
代码结构都是由前导代码和主体代码组成,完全沿用了LaTeX的文档环境与基本命令。因此,使用
Beamer制作幻灯片也有一个缺点,那就是必须掌握LaTeX制作文档的基础。

在呈现方式上,Beamer制作的幻灯片最终会在LaTeX中被编译成PDF文档,在不同的系统上打开幻灯
片时不存在不兼容等问题。在功能上,使用Beamer制作幻灯片时,我们可以对常规文本、公式、列
表、图表甚至动画效果、视觉效果和主题样式等进行调整,最终达到想要的视觉效果。

事实上,LaTeX中用于制作演示文稿的工具并非只有Beamer,但相比其他工具,Beamer具有如下优点:
\begin{itemize}
    \item 拥有海量的模板和丰富的主题样式,且使用方便;
    \item 能满足制作幻灯片的功能性需求,从创建标题、文本和段落到插入图表、参考文献等操作,且与常规文档中的使用规则几乎一致;
    \item 使用方式简单,在主体代码中使用frame环境就能创建一页幻灯片。
\end{itemize}

本章主要包括以下内容:Beamer的基本使用方式、在演示稿中添加动画效果、添加文本框等框元素、
设置主题样式、插入程序源代码、添加参考文献、插入表格、插入与调整图片。

\section{基本介绍}

Beamer是一款灵活的幻灯片制作工具,我们可以在LaTeX中将它作为一种文档类型进行使用。本节主
要介绍Beamer的基本使用方式,包括创建幻灯片、创建章节、生成目录等操作。

\subsection{Beamer简介}

在上述章节中,我们主要介绍了LaTeX中比较常用的文档类型article,可用于创建期刊论文、技术
报告等。本章中我们将介绍另一种文档类型:beamer。Beamer的开发者Till Tantau说,“BEAMER is a LATEX class for creating presentations”,显然,Beamer是一种用于制作演示文稿或者幻灯片的文档类型。

从使用角度来说,beamer文档类型和book、article等文档类型一样,都是在以\emph{.tex}为拓展名的文件
上编写程序和文档内容,然后再通过编译生成PDF文档。当然,Beamer也兼具常用演示文稿如PowerPoint
的主要功能,可以自行设置动态效果、甚至使用主题样式修改幻灯片的外观。

与其他文档类型相似的是,beamer文档类型中拥有很多视觉效果极好的模板,这些模版已经设置好了
特定的主题样式,有时候甚至只需要加入创作内容即可得到心仪的幻灯片。使用Beamer制作幻灯片时,
我们可以体验LaTeX排版论文的几乎所有优点,公式排版、图表排版、参考文献设置等也非常便捷,有
时候甚至可以将常规文档中的内容直接复制到Beamer文档类型中,稍加调整便能得到样式合适的幻灯
片。另外,我们也可以根据需要,在前导代码中使用全局设置调整幻灯片的主题样式、颜色主题、字体主题等。

使用beamer制作幻灯片仍然遵循着LaTeX的一般使用方法,代码结构分为前导代码和主体代码,前导
代码除了申明文档类型为beamer外,即\texttt{\textbackslash{}documentclass\{beamer\}},
调用宏包等与常规文档的制作基本是一致的。

\emph{【例】}使用beamer文档类型创建一个简单的幻灯片:
\begin{lstlisting}[language=TeX]
    \documentclass{beamer}

    \title{A Simple Beamer Example}
    \author{Author's Name}
    \institute{Author's Institute}
    \date{\today} 

    \begin{document}

    \frame{\titlepage}

    \end{document}
\end{lstlisting}

在例子中,\texttt{\textbackslash{}title\{\}}、\texttt{\textbackslash{}author\{\}}和
\texttt{\textbackslash{}date\{\}}这几个命令分别对应着标题、作者以及日期,一般放在标题
页,如果想在幻灯片首页显示这些信息,可以在使用\texttt{\textbackslash{}frame\{\textbackslash{}titlepage\}}
命令新建标题页。

总结来说,标题及作者信息对应的特定命令包括:
\begin{itemize}
    \item 标题:对应的命令为\texttt{\textbackslash{}title[A]\{B\}},其中,位置A一般填写的是简化标题,而位置B则填写的是完整标题,这里的完整标题有时候可能会很长。
    \item 副标题:对应的命令为\texttt{\textbackslash{}subtitle[A]\{B\}},其中,位置A一般填写的是简化副标题,而位置B则填写的是完整副标题,这里的完整副标题有时候也可能会很长。
    \item 作者:对应的命令为\texttt{\textbackslash{}author[A]\{B\}},用法类似。
    \item 日期:对应的命令为\texttt{\textbackslash{}date[A]\{B\}},用法类似。
    \item 单位:对应的命令为\texttt{\textbackslash{}institution[A]\{B\}},用法类似。
\end{itemize}

我们知道,在常规文档article中,申明文档类型时可以指定正文字体大小,在文档类型的申明语句
\texttt{\textbackslash{}documentclass\{beamer\}}中,我们也可以通过特定选项调整幻灯片
内容的字体大小,一般默认为11pt,我们也可以根据需要使用8pt、9pt、10pt、12pt、14pt、17pt、20pt
字体大小,例如使用\texttt{\textbackslash{}documentclass[12pt]\{beamer\}}可以将字体大小设置为12pt。

制作幻灯片时,有时候为了达到特定的投影效果,会设置幻灯片的长宽比例,比较常用的两种长宽比
例分别为4:3和16:9。一般来说,Beamer制作出来的幻灯片默认大小为长128毫米、宽96毫米,对应
着默认的长宽比例4:3,有时候,我们也可以根据需要将幻灯片的长宽比例调整为16:9、14:9、5:4
甚至3:2。

\emph{【例】}使用beamer文档类型创建一个简单的幻灯片,将幻灯片的长宽比例调整为16:9:
\begin{lstlisting}[language=TeX]
    \documentclass[aspectratio = 169]{beamer}

    \title{A Simple Beamer Example}
    \author{Author's Name}
    \institute{Author's Institute}
    \date{\today} 

    \begin{document}

    \frame{\titlepage}

    \end{document}
\end{lstlisting}

在例子中,选项aspectratio对应着长宽比例,数字169对应着长宽比例16:9,类似地,149、54、32
分别对应着长宽比例14:9、5:4、3:2。

\subsection{创建幻灯片}

frame这个词在计算机编程中非常常见,这一英文单词的字面意思可以翻译为“帧”,假如我们将幻灯
片视作“连环画”,是由一页一页单独的幻灯片组成,那么每一页幻灯片则对应着连环画中的帧。使用
Beamer制作幻灯片时,幻灯片就是用frame环境创建出来的,然而,有时候为了让幻灯片产生动画视
觉效果,Beamer中的帧(即frame)与每页幻灯片并非严格意义上的一一对应。

在beamer文档类型中,制作幻灯片的环境一般为frame。在document环境构成的主体代码中,一个frame
环境一般对应着一页幻灯片。

每张幻灯片一般都有一个标题,有时也会有一个副标题。若要创建标题和副标题,用户可以通过使用
\texttt{\textbackslash{}begin{frame}\{\}\{\}}的命令格式,其中第一、二个\{\}中分别为
幻灯片的标题和副标题;此外,用户也可以通过在frame环境中,使用\texttt{\textbackslash{}frametitle\{\}}
和\texttt{\textbackslash{}framesubtitle\{\}}命令分别创建标题和副标题。由此创建的标题
和副标题一般位于幻灯片的顶部,标题相对于副标题字体稍大一点。

实际上,Beamer与其他文档类型并没有特别大的差异,常规文档中的基本列表环境都可以在Beamer中
使用,包括:有序列表环境enumerate、无序列表环境itemize以及解释性列表环境description。

\emph{【例】}使用beamer文档类型创建一个简单的幻灯片:
\begin{lstlisting}[language=TeX]
    \documentclass{beamer}
    \usefonttheme{professionalfonts}

    \begin{document}

    \begin{frame}
    \frametitle{Parent function}
    \framesubtitle{A short list}

    Please check out the following parent function list.
    \begin{enumerate}
    \item $y=x$
    \item $y=|x|$
    \item $y=x^{2}$
    \item $y=x^{3}$
    \item $y=x^{b}$
    \end{enumerate}

    \end{frame}

    \end{document}
\end{lstlisting}

有时为了简化代码,也可以直接用\texttt{\textbackslash{}frame\{\}}命令取代frame环境囊括
幻灯片内容。

\emph{【例】}使用beamer文档类型中的frame简化环境命令创建一个简单的幻灯片:
\begin{lstlisting}[language=TeX]
    \documentclass{beamer}
    \usefonttheme{professionalfonts}

    \begin{document}

    \frame{
    \frametitle{Parent function}
    \framesubtitle{A short list}

    Please check out the following parent function list.
    \begin{enumerate}
    \item $y=x$
    \item $y=|x|$
    \item $y=x^{2}$
    \item $y=x^{3}$
    \item $y=x^{b}$
    \end{enumerate}

    \end{document}
\end{lstlisting}

使用Beamer制作幻灯片时,幻灯片内容会在标题下方自动居中对齐,如果想调整对其方式,可以在frame
环境中设置参数,具体而言,有以下几种:
\begin{itemize}
    \item [c] 居中对齐,字母c对应着英文单词center的首字母,一般而言,[c]作为默认参数,无需专门设置;
    \item [t] 让幻灯片内容进行顶部对齐,其中,字母t对应着英文单词top的首字母;
    \item [b] 让幻灯片内容进行底部对齐,其中,字母b对应着英文单词bottom的首字母。
\end{itemize}

\emph{【例】}使用beamer文档类型中的frame环境创建一个简单的幻灯片,并让幻灯片内容进行顶部对齐:
\begin{lstlisting}[language=TeX]
    \documentclass{beamer}
    \usefonttheme{professionalfonts}

    \begin{document}

    \begin{frame}[t]
    \frametitle{Parent function}
    \framesubtitle{A short list}

    Please check out the following parent function list.
    \begin{enumerate}
    \item $y=x$
    \item $y=|x|$
    \item $y=x^{2}$
    \item $y=x^{3}$
    \item $y=x^{b}$
    \end{enumerate}

    \end{frame}

    \end{document}
\end{lstlisting}

上面例子介绍了如何创建单页幻灯片,类似地,可以使用多个frame环境制作多页幻灯片。

\emph{【例】}使用beamer文档类型中的frame环境创建一个多页的幻灯片:
\begin{lstlisting}[language=TeX]
    \documentclass{beamer}

    \title{The title}
    \subtitle{The subtitle}
    \author{Author's name}

    \begin{document}

    \begin{frame}
        \titlepage % 创建标题页
    \end{frame}

    \begin{frame}
    \frametitle{Frame title}
    The body of the frame.
    \end{frame}

    \end{document}
\end{lstlisting}

