\chapter{文本编辑}
文本编辑是组成科技论文或科技报告的重要主要内容。而对于论文或报告中,简洁美观的文本对于阅
读者的理解以及感受是非常重要的。因此,我们需要认真学习在Latex中如何编辑出清晰明了的文本
内容。

文本编辑的内容一般包括文本章节的设定、文本段落的编辑、文字格式的编辑、列表的创建、页眉页
脚和脚注的创建。其中文本段落的编辑又包括段落首行缩进、段行间距、段落对齐方式的调整等,而
文字的编辑主要包括字体样式的调整、字体大小的调整、字体颜色的调整、字体本身的设定、字体中
的下划线和删除线、以及一些特殊字符的书写等内容。

列表的创建是方便读者对文本进行阅读,而这一类文本中的内容一般处于并列关系。合适的列表可以
方便读者知道文本的框架和内容关系。列表内容主要包括无序列表、排序列表、阐述性列表和自定义
列表格式。对于文本编辑,页眉页脚和脚注是很重要的,页眉一般可以用于显示题目或者章节名称等
提醒内容,而页脚可以显示文本页码等重要信息,脚注可以用于备注一些重要特别内容,如作者个人
信息以及项目资助信息等。本章将详细介绍文本编辑的相关内容,学好文本编辑有助于我们完成一个
阅读性更好的高质量文档。

\section{创建标题部分、摘要及关键词}
文档主体代码是指位于document环境的部分。在文档正文章节内容及目录之前,一般先创建标题部
分(包括文档标题、作者和日期)、摘要、以及关键词信息,这也是文档主体代码中最开始的部分。
下面分别介绍这部分的创建过程。

\subsection{创建标题部分}
\begin{itemize}
      \item 使用\verb|\|title\{\}命令设置文档标题,对于较长的文档标题,
            可以使用\verb|\\|对标题内容进行分行。
      \item 使用\verb|\|author\{\}命令设置作者,如果有多个作者,作者之间可以使
            用\verb|\|and进行分隔。
      \item 使用\verb|\|date\{\}命令设置日期信息,在实际使用时,有时需要省略日期
            信息,那么在\{\}中不写任何内容即可。如果想要使用默认值(当前日期),则应使用\verb|\|date命令。
      \item 使用\verb|\|maketitle命令完成标题部分的创建。
\end{itemize}

仅仅执行上述三行语句无法在文档编译时生成标题部分,还必须在之后加上\verb|\|maketitle
命令,表示对标题部分内容进行排版才能真正实现标题部分的创建。

\subsection{创建摘要及关键词}
在LaTeX中,使用\emph{abstract}环境撰写文档摘要部分,并在其后使用\verb|\|textbf\{\}
命令设置文档关键词。

\section{创建章节}
在制作文档时,不管是学术论文还是书籍,都需要创建章节来优化文章的层次结构。LaTeX提供了不
同层次章节的创建命令,从高到低包括:使用\verb|\|part\{\}命令创建不同篇、使用\verb|\|chapter\{\}
命令创建不同章、使用\verb|\|section\{\}命令创建一级节、使用\verb|\|subsection\{\}命令
创建二级节、以及使用\verb|\|subsubsection\{\}命令创建三级节。各篇章节的标题填写在\{\}中。

对于article类型文档,可以调用上述除了\verb|\|chapter\{\}命令之外的其他章节命令。而在
book、report类型文档中,则可以调用上述所有章节命令。

在撰写学术论文时,不同出版商所要求的章节标题格式可能是不一样的。因此,我们需要根据不同需
要调整章节标题的格式。

\begin{itemize}
      \item 自动编号与取消自动编号:在LaTeX中使用上述命令创建章节时,默认会对各章节进行自
            动编号。如果用户想要对某章节取消自动编号,只需要在章节命令后加上星号即可,如\verb|\|section*\{\}
            命令。
      \item 设置章节自动编号深度:用户也可以通过在导言区使用\verb|\|setcounter\{secnumdepth\}\{\}
            计数器命令设置章节自动编号深度,从而达到批量取消自动编号的效果。在\{\}中填写编号深度
            值,编号深度值从0开始设置,表示章节自动编号深入层次:
            \begin{itemize}
                  \item \verb|\|setcounter\{secnumdepth\}\{0\}表示自动编号章节层次仅包括\verb|\|part和\verb|\|chapter;
                  \item \verb|\|setcounter\{secnumdepth\}\{1\}表示自动编号层次深入到\verb|\|section级;
                  \item \verb|\|setcounter\{secnumdepth\}\{2\}表示自动编号层次深入到\verb|\|subsection级;
                  \item \verb|\|setcounter\{secnumdepth\}\{3\}表示自动编号层次深入到\verb|\|subsubsection级,默认值。
            \end{itemize}
      \item 改变字体样式:对于改变字体样式,我们需要使用\emph{titlesec}宏包,使用该宏包中
            的命令\verb|\|titleformat*\{\}\{\}来改变字体样式。
\end{itemize}

当然,有时我们也会遇到一些出版商要求我们将标题居中显示,在LaTeX的article文档类型中,我们
可以调用\emph{sectsty}宏包,并用到\verb|\|sectionfont\{\verb|\|centering\}使标题居中。

\section{生成目录}
章节目录一般在摘要之后创建,使用\verb|\|tableofcontents命令即可。命令放在哪里,就会在
哪里自动创建一个章节目录。默认情况下,该命令会根据用户定义的篇章节标题生成文章目录,目录
中将包含\verb|\|subsubsection及其更高层次的结构标题。但对于带星号的章节命令,其章节标
题不会出现在目录中。

根据需要,用户可以对目录格式进行各项调整。

\subsection{调整章节层次深度}
在前一节中我们介绍了使用\verb|\|setcounter\{secnumdepth\}\{\}计数器命令调整章节自动编
号深度,类似地,我们可以通过在导言区使用\verb|\|setcounter\{tocdepth\}\{\}命令指定目录
中的章节层次深度。
\begin{itemize}
      \item \verb|\|setcounter\{tocdepth\}\{0\},目录层次仅包括part和chapter;
      \item \verb|\|setcounter\{tocdepth\}\{1\},设置目录层次深入到section级;
      \item \verb|\|setcounter\{tocdepth\}\{2\},设置目录层次深入到subsection级;
      \item \verb|\|setcounter\{tocdepth\}\{3\},设置目录层次深入到subsubsection级,默认值;
\end{itemize}

上面的语句可以为所有章节指定了相同的目录层次深度。此外,我们也可以为每个章节设置不同的目录
层次,具体是通过在每个章节创建命令前,使用\verb|\|addtocontents\{toc\}{\verb|\|setcounter\{tocdepth\}\{\}\}
命令为该章节指定目录层次深度。

\subsection{给目录设置别名}
对于章节标题特别长的情况,直接在目录中显示完整标题显然可视化效果不佳,因此需要为长章节标题
设置一个比较短的“目录别名”。通过这种设置,在正文中可以显示完整标题,而在目录中将显示“短标题”。
为此,只需要在章节创建命令中添加目录别名选项即可。例:\verb|\|section[FS]\{First Section\}

\subsection{给目录设置链接}
如果想要为目录中的章节引用添加链接,使得点击链接后就能跳转到相应章节所在页面,那么只需要
在导言区调用\emph{hyperref}宏包。如果设置colorlinks=true选项,此时文档中章节引用及其
它交叉引用均会被自动添加链接(添加了链接的引用将显示为红色)。

\section{编辑段落}
\subsection{段落首行缩进}
许多出版社要求文章段落必须首行缩进,若想调整段落首行缩进的距离,可以使用\verb|\|setlength\{\verb|\|parindent\}\{长度\}
命令,在\{长度\}处填写需要设置的距离即可。例:\verb|\|setlength\{\verb|\|parindent\}\{2em\}

当然,如果不想让段落自动首行缩进, 在段落前使用命令\verb|\|noindent即可。

需要注意的是,在段落设置在章节后面时,每一节后的第一段默认是不缩进的,为了使第一段向其他
段一样缩进,可以在段落前使用\verb|\|hspace*\{\verb|\|parindent\}命令,也可以在源文件的
前导代码中直接调用宏包\verb|\|usepackage\{indentfirst\}。

\subsection{段落间距调整}
在使用LaTeX排版时,有时为了使段落与段落之间的区别更加明显,我们可以在段落之间设置一定的
间距,最简单的方式是使用\verb|\|smallskip、\verb|\|medskip和\verb|\|bigskip等命令。

\begin{tcolorbox}[colback=red!5!white, colframe=red!50!black,
            title=设置段落间距的几种方法]
      https://latex.org/forum/viewtopic.php?f=44\&t=6934
      \tcblower
      https://tex.stackexchange.com/questions/41476/lengths-and-when-to-use-them
\end{tcolorbox}

\subsection{段落添加文本框}
有时因为文档没有图全都是文字,版面显得极其单调。如果想让版面有所变化,可以通过给文字加边
框来实现对段落文本新增边框。在LaTeX中,我们可以使用\verb|\|fbox\{\}命令对文本新增边框。

\begin{tcolorbox}[colback=red!5!white, colframe=red!50!black,
            title=How to put a box around multiple lines?]
      https://latex.org/forum/viewtopic.php?f=44\&t=4117
\end{tcolorbox}

\subsection{段落对齐方式调整}
LaTeX默认的对齐方式是两端对齐,有时在进行文档排版的过程中,我们为了突出某一段落的内容,
会选择将其居中显示,在LaTeX中,我们可以使用center环境对文本进行居中对齐。另外还有一些出
版商要求文档是左对齐或者右对齐,这时我们同样可以使用\emph{flushleft}环境和\emph{flushright}
环境将文档设置为左对齐或右对齐。