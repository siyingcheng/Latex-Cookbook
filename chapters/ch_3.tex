\chapter{文本编辑}
文本编辑是组成科技论文或科技报告的重要主要内容。而对于论文或报告中,简洁美观的文本对于阅
读者的理解以及感受是非常重要的。因此,我们需要认真学习在Latex中如何编辑出清晰明了的文本
内容。

文本编辑的内容一般包括文本章节的设定、文本段落的编辑、文字格式的编辑、列表的创建、页眉页
脚和脚注的创建。其中文本段落的编辑又包括段落首行缩进、段行间距、段落对齐方式的调整等,而
文字的编辑主要包括字体样式的调整、字体大小的调整、字体颜色的调整、字体本身的设定、字体中
的下划线和删除线、以及一些特殊字符的书写等内容。

列表的创建是方便读者对文本进行阅读,而这一类文本中的内容一般处于并列关系。合适的列表可以
方便读者知道文本的框架和内容关系。列表内容主要包括无序列表、排序列表、阐述性列表和自定义
列表格式。对于文本编辑,页眉页脚和脚注是很重要的,页眉一般可以用于显示题目或者章节名称等
提醒内容,而页脚可以显示文本页码等重要信息,脚注可以用于备注一些重要特别内容,如作者个人
信息以及项目资助信息等。本章将详细介绍文本编辑的相关内容,学好文本编辑有助于我们完成一个
阅读性更好的高质量文档。

\section{创建标题部分、摘要及关键词}
文档主体代码是指位于document环境的部分。在文档正文章节内容及目录之前,一般先创建标题部
分(包括文档标题、作者和日期)、摘要、以及关键词信息,这也是文档主体代码中最开始的部分。
下面分别介绍这部分的创建过程。

\subsection{创建标题部分}
\begin{itemize}
      \item 使用\verb|\|title\{\}命令设置文档标题,对于较长的文档标题,
            可以使用\verb|\\|对标题内容进行分行。
      \item 使用\verb|\|author\{\}命令设置作者,如果有多个作者,作者之间可以使
            用\verb|\|and进行分隔。
      \item 使用\verb|\|date\{\}命令设置日期信息,在实际使用时,有时需要省略日期
            信息,那么在\{\}中不写任何内容即可。如果想要使用默认值(当前日期),则应使用\verb|\|date命令。
      \item 使用\verb|\|maketitle命令完成标题部分的创建。
\end{itemize}

仅仅执行上述三行语句无法在文档编译时生成标题部分,还必须在之后加上\verb|\|maketitle
命令,表示对标题部分内容进行排版才能真正实现标题部分的创建。

\subsection{创建摘要及关键词}
在LaTeX中,使用\emph{abstract}环境撰写文档摘要部分,并在其后使用\verb|\|textbf\{\}
命令设置文档关键词。

\section{创建章节}
在制作文档时,不管是学术论文还是书籍,都需要创建章节来优化文章的层次结构。LaTeX提供了不
同层次章节的创建命令,从高到低包括:使用\verb|\|part\{\}命令创建不同篇、使用\verb|\|chapter\{\}
命令创建不同章、使用\verb|\|section\{\}命令创建一级节、使用\verb|\|subsection\{\}命令
创建二级节、以及使用\verb|\|subsubsection\{\}命令创建三级节。各篇章节的标题填写在\{\}中。

对于article类型文档,可以调用上述除了\verb|\|chapter\{\}命令之外的其他章节命令。而在
book、report类型文档中,则可以调用上述所有章节命令。

在撰写学术论文时,不同出版商所要求的章节标题格式可能是不一样的。因此,我们需要根据不同需
要调整章节标题的格式。

\begin{itemize}
      \item 自动编号与取消自动编号:在LaTeX中使用上述命令创建章节时,默认会对各章节进行自
            动编号。如果用户想要对某章节取消自动编号,只需要在章节命令后加上星号即可,如\verb|\|section*\{\}
            命令。
      \item 设置章节自动编号深度:用户也可以通过在导言区使用\verb|\|setcounter\{secnumdepth\}\{\}
            计数器命令设置章节自动编号深度,从而达到批量取消自动编号的效果。在\{\}中填写编号深度
            值,编号深度值从0开始设置,表示章节自动编号深入层次:
            \begin{itemize}
                  \item \verb|\|setcounter\{secnumdepth\}\{0\}表示自动编号章节层次仅包括\verb|\|part和\verb|\|chapter;
                  \item \verb|\|setcounter\{secnumdepth\}\{1\}表示自动编号层次深入到\verb|\|section级;
                  \item \verb|\|setcounter\{secnumdepth\}\{2\}表示自动编号层次深入到\verb|\|subsection级;
                  \item \verb|\|setcounter\{secnumdepth\}\{3\}表示自动编号层次深入到\verb|\|subsubsection级,默认值。
            \end{itemize}
      \item 改变字体样式:对于改变字体样式,我们需要使用\emph{titlesec}宏包,使用该宏包中
            的命令\verb|\|titleformat*\{\}\{\}来改变字体样式。
\end{itemize}

当然,有时我们也会遇到一些出版商要求我们将标题居中显示,在LaTeX的article文档类型中,我们
可以调用\emph{sectsty}宏包,并用到\verb|\|sectionfont\{\verb|\|centering\}使标题居中。

\section{生成目录}
章节目录一般在摘要之后创建,使用\verb|\|tableofcontents命令即可。命令放在哪里,就会在
哪里自动创建一个章节目录。默认情况下,该命令会根据用户定义的篇章节标题生成文章目录,目录
中将包含\verb|\|subsubsection及其更高层次的结构标题。但对于带星号的章节命令,其章节标
题不会出现在目录中。

根据需要,用户可以对目录格式进行各项调整。

\subsection{调整章节层次深度}
在前一节中我们介绍了使用\verb|\|setcounter\{secnumdepth\}\{\}计数器命令调整章节自动编
号深度,类似地,我们可以通过在导言区使用\verb|\|setcounter\{tocdepth\}\{\}命令指定目录
中的章节层次深度。
\begin{itemize}
      \item \verb|\|setcounter\{tocdepth\}\{0\},目录层次仅包括part和chapter;
      \item \verb|\|setcounter\{tocdepth\}\{1\},设置目录层次深入到section级;
      \item \verb|\|setcounter\{tocdepth\}\{2\},设置目录层次深入到subsection级;
      \item \verb|\|setcounter\{tocdepth\}\{3\},设置目录层次深入到subsubsection级,默认值;
\end{itemize}

上面的语句可以为所有章节指定了相同的目录层次深度。此外,我们也可以为每个章节设置不同的目录
层次,具体是通过在每个章节创建命令前,使用\verb|\|addtocontents\{toc\}{\verb|\|setcounter\{tocdepth\}\{\}\}
命令为该章节指定目录层次深度。

\subsection{给目录设置别名}
对于章节标题特别长的情况,直接在目录中显示完整标题显然可视化效果不佳,因此需要为长章节标题
设置一个比较短的“目录别名”。通过这种设置,在正文中可以显示完整标题,而在目录中将显示“短标题”。
为此,只需要在章节创建命令中添加目录别名选项即可。例:\verb|\|section[FS]\{First Section\}

\subsection{给目录设置链接}
如果想要为目录中的章节引用添加链接,使得点击链接后就能跳转到相应章节所在页面,那么只需要
在导言区调用\emph{hyperref}宏包。如果设置colorlinks=true选项,此时文档中章节引用及其
它交叉引用均会被自动添加链接(添加了链接的引用将显示为红色)。

\section{编辑段落}
\subsection{段落首行缩进}
许多出版社要求文章段落必须首行缩进,若想调整段落首行缩进的距离,可以使用\verb|\|setlength\{\verb|\|parindent\}\{长度\}
命令,在\{长度\}处填写需要设置的距离即可。例:\verb|\|setlength\{\verb|\|parindent\}\{2em\}

当然,如果不想让段落自动首行缩进, 在段落前使用命令\verb|\|noindent即可。

需要注意的是,在段落设置在章节后面时,每一节后的第一段默认是不缩进的,为了使第一段向其他
段一样缩进,可以在段落前使用\verb|\|hspace*\{\verb|\|parindent\}命令,也可以在源文件的
前导代码中直接调用宏包\verb|\|usepackage\{indentfirst\}。

\subsection{段落间距调整}
在使用LaTeX排版时,有时为了使段落与段落之间的区别更加明显,我们可以在段落之间设置一定的
间距,最简单的方式是使用\verb|\|smallskip、\verb|\|medskip和\verb|\|bigskip等命令。

\begin{tcolorbox}[colback=red!5!white, colframe=red!50!black,
            title=设置段落间距的几种方法]
      https://latex.org/forum/viewtopic.php?f=44\&t=6934
      \tcblower
      https://tex.stackexchange.com/questions/41476/lengths-and-when-to-use-them
\end{tcolorbox}

\subsection{段落添加文本框}
有时因为文档没有图全都是文字,版面显得极其单调。如果想让版面有所变化,可以通过给文字加边
框来实现对段落文本新增边框。在LaTeX中,我们可以使用\verb|\|fbox\{\}命令对文本新增边框。

\begin{tcolorbox}[colback=red!5!white, colframe=red!50!black,
            title=How to put a box around multiple lines?]
      https://latex.org/forum/viewtopic.php?f=44\&t=4117
\end{tcolorbox}

\subsection{段落对齐方式调整}
LaTeX默认的对齐方式是两端对齐,有时在进行文档排版的过程中,我们为了突出某一段落的内容,
会选择将其居中显示,在LaTeX中,我们可以使用\emph{center}环境对文本进行居中对齐。另外还有一些出
版商要求文档是左对齐或者右对齐,这时我们同样可以使用\emph{flushleft}环境和\emph{flushright}
环境将文档设置为左对齐或右对齐。

\section{文字编辑}
文字编辑是制作文档非常重要的一部分,主要关注如何调整字体样式、字体设置、增加下划线、突出
文字、调整字体大小、调整对齐格式等内容。

\subsection{调整字体样式}
调整文字的样式有很多对应的命令,这些命令包括\verb|\|textit、\verb|\|textbf、\verb|\|textsc、\verb|\|texttt,
在使用的过程中,需要用到花括号\{\}。具体而言,\verb|\|textit对应着斜体字,\verb|\|textbf对应着粗体字,
\verb|\|textsc对应着小型大写字母,\verb|\|texttt对应着打印机字体(即等宽字体)。

除了这几种字体样式,有时候,如果想对文本中的英文字母进行全部大写,可用\verb|\|uppercase和
\verb|\|MakeUppercase两个命令中的任意一个。

一般而言,当我们需要对段落、句子、关键词等做特殊标记时,往往会用到上述几种字体样式,其中,
打字机字体主要用于代码的排版,有时候,如果需要添加一个网站,通常也会选用打字机字体对文本进行
突出,例如\verb|\|texttt\{https://www.overleaf.com\}。

\subsection{调整字体大小}
字体大小的设置一方面可以在申明文档类型的命令\verb|\|documentclass[]\{\}中指定具体的字体
大小(如11pt、12pt)来实现,另一方面也可以通过一些简单的命令来调整。
\begin{lstlisting}[language=TeX]
      \documentclass[12pt]{article}
      \begin{document}

      Produce {\tiny tiny word}

      Produce {\scriptsize script size word}

      Produce {\footnotesize footnote size word}

      Produce {\normalsize normal size word}

      Produce {\large large word}

      Produce {\Large Large word}

      Produce {\LARGE LARGE word}

      Produce {\huge huge word}

      Produce {\Huge Huge word}

      \end{document}
\end{lstlisting}

在这里,这些命令对应的字体依次是从小到大。当然,这些命令也有另外一种使用方法,以
\verb|\|large、\verb|\|Large、\verb|\|LARGE为例,我们可以使用\verb|\|begin\{\} \verb|\|end\{\}
语句来实现对字体的加大:
\begin{lstlisting}[language=TeX]
      \documentclass[12pt]{article}
      \begin{document}

      Produce \begin{large}large word\end{large}

      Produce \begin{Large}large word\end{Large}

      Produce \begin{LARGE}large word\end{LARGE}

      \end{document}
\end{lstlisting}

\subsection{调整字体颜色}
一般而言,文本默认的颜色是黑色,但有时候,我们可以根据需要改变字体的颜色,这通过LaTeX一些
拓展的宏包就可以实现,例如\emph{xcolor}。

使用颜色宏包时,我们也可以根据需要自定义颜色,相应的命令为\verb|\|definecolor\{A\}\{B\}\{C\},其中
位置A是颜色标签,位置B是制定颜色系统为RGB(英文缩写RGB是红色、绿色和蓝色三种颜色的英文
单词首字母),位置C是具体的RGB数值。
\begin{lstlisting}[language=TeX]
      \documentclass[12pt]{article}

      \usepackage{color}
      \definecolor{kugreen}{RGB}{50, 93, 61}
      \definecolor{kugreenlys}{RGB}{132, 158, 139}
      \definecolor{kugreenlyslys}{RGB}{173, 190, 177}
      \definecolor{kugreenlyslyslys}{RGB}{214, 223, 216}

      \begin{document}

      This is a simple example for using \LaTeX.

      {\color{kugreen}This is a simple example for using \LaTeX.}

      {\color{kugreenlys}This is a simple example for using \LaTeX.}

      {\color{kugreenlyslys}This is a simple example for using \LaTeX.}

      {\color{kugreenlyslyslys}This is a simple example for using \LaTeX.}

      \end{document}
\end{lstlisting}

\subsection{字体设置}
不管是英文还是中文,我们都会越到各种各样的字体,因此,使用LaTeX编译出想要的字体对于整个
文档也非常重要。对于英文文档的编译,一般会用宏包fontspec设置具体的字体,调用格式为\verb|\|usepackage\{fontspec\},
需要说明的是,这个宏包只能设置英文的字体。例如:
\begin{lstlisting}[language=TeX]
      \setmainfont{Times New Roman}
      \setsansfont{DejaVu Sans}
      \setmonofont{Latin Modern Mono}
      \setsansfont{[foo.ttf]}   
\end{lstlisting}

如果文档输入的是中文,首先需要申明文档类型为ctex中的ctexart、ctexrep之类的。

在LaTeX中,编译文档一般默认的英文字体为Computer Modern,如果要将其调整为其他特定类型的
字体,可以在前导代码中使用各种字体对应的工具包。

\begin{tcolorbox}[colback=red!5!white, colframe=red!50!black,
            title=更多字体设置参考]
      https://www.overleaf.com/learn/latex/Font\_typefaces
\end{tcolorbox}

\subsection{下划线与删除线}
有时候,为了突出特定的文本,我们也会使用到各种下划线。最常用的下划线命令是\verb|\|underline,
然而,这个命令存在一个缺陷,即当文本过长,超过页面宽度时,下划线不会自动跳到下一行,因此,
我们需要用到一个叫\emph{ulem}的宏包,使用这个宏包中的命令\verb|\|uline可以增加单下划线,
使用\verb|\|uuline可以增加双下划线,而使用\verb|\|uwave则可以增加波浪线。

\begin{lstlisting}[language=TeX]
      \documentclass[12pt]{article}
      \usepackage{ulem}
      \begin{document}

      Generate \underline{underlined} text. \\     % 生成带下划线的文本(使用\underline命令)

      Generate \uline{underlined} text. \\         % 生成单下划线的文本(使用\uline命令)

      Generate \uuline{double underlined} text. \\ % 生成单下划线的文本

      Generate \uwave{wavy underlined} text. \\    % 生成波浪线的文本

      \end{document}
\end{lstlisting}

删除线是文字中间划出的线段,常见于文档的审阅。在LaTeX中,我们可以使用宏包\emph{soul}中的\verb|\|st
命令生成删除线。

下划线在LaTeX环境中属于特殊字符,如果需要在文本中使用下划线,则需要家上反斜线进行转义,例:
\begin{lstlisting}[language=TeX]
      \documentclass[12pt]{article}
      \begin{document}

      This\_is\_text\_with\_underscores.

      \end{document}
\end{lstlisting}

\subsection{特殊字符}
在LaTeX中,有很多特殊字符的编译需要遵循一定的规则,例如:
\begin{itemize}
      \item 反斜杠 (backslash) 符号是LaTeX中定义和使用各类命令的基本符号,如果要在文档
            中编译出反斜杠,可使用\verb|\|textbackslash;
      \item 百分号通常用于注释代码,如果要在文档中编译出百分号,可使用\verb|\|\%;
      \item 美元符号通常用于书写公式,如果要在文档中编译出美元符号,可使用\verb|\|\$。
\end{itemize}

带圆圈数字可用于各类编号,我们可以根据需要插入这种特殊符号。在LaTeX中,比较常用的一种生
成带圆圈数字的方法是使用宏包\emph{pifont},在前导代码中申明使用宏包,即\verb|\|usepackage\{pifont\},
根据工宏包所提供的命令\verb|\|ding\{\}可以生成从1到10的带圆圈数字,且圆圈样式也各异。

\emph{【例】}使用pifont宏包中的命令生成从1到10的带圆圈数字:
\begin{lstlisting}[language=TeX]
      \documentclass[12pt]{article}
      \usepackage{pifont}
      \begin{document}

      How to write a number in a circle? \\
      Type 1: \ding{172}-\ding{173}-\ding{174}-\ding{175}-\ding{176}-\ding{177}-\ding{178}-\ding{179}-\ding{180}-\ding{181} \\     % 样式1是从172开始
      Type 2: \ding{182}-\ding{183}-\ding{184}-\ding{185}-\ding{186}-\ding{187}-\ding{188}-\ding{189}-\ding{190}-\ding{191} \\     % 样式2是从182开始
      Type 3: \ding{192}-\ding{193}-\ding{194}-\ding{195}-\ding{196}-\ding{197}-\ding{198}-\ding{199}-\ding{200}-\ding{201} \\     % 样式3是从192开始
      Type 4: \ding{202}-\ding{203}-\ding{204}-\ding{205}-\ding{206}-\ding{207}-\ding{208}-\ding{209}-\ding{210}-\ding{211} \\     % 样式4是从202开始

      \end{document}
\end{lstlisting}